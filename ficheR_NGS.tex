\documentclass[11, a4paper, landscape]{article}
\usepackage[utf8]{inputenc}
\usepackage[T1]{fontenc} 
\usepackage[frenchb]{babel} 
\usepackage{amsmath}
\usepackage{amsfonts}
\usepackage{amssymb}
\usepackage[pdftex]{graphicx}
\usepackage{array}
\usepackage{multicol}
\usepackage{fullpage}
\usepackage{textcomp}
\usepackage{lscape}  
\usepackage{subfigure}        %%%%%format paysage avec\begin{landscape}
\usepackage{fancyhdr}
\usepackage[absolute]{textpos} 
\usepackage[usenames,dvipsnames]{xcolor}

\usepackage{multicol}
\setlength{\columnseprule}{0.5pt} %%%% ligne de séparation entre les colonnes
\setlength{\columnsep}{25pt} %%%% séparation entre colonnes

\usepackage{geometry}
\geometry{left=0.8cm, bottom=1.3cm, top=0.5cm, right=0.8cm}

\usepackage{makeidx} 
\usepackage[hidelinks]{hyperref}
\colorlet{DarkMidnightBlue}{MidnightBlue!70!black}
\hypersetup{linkcolor=DarkMidnightBlue, colorlinks=TRUE, urlcolor=MidnightBlue }

\hypersetup{
pdftitle = {Fiche de commande R pour l'analyse de communauté via des méthodes NGS},
pdfsubject = {Fiche de commande R pour les NGS},
pdfkeywords = {R, ecologie, NGS},
pdfauthor = {\textcopyright\ Adrien Taudiere},
pdfcreator = {\LaTeX\ with package \flqq hyperref\frqq},
}

%%%%%%%%%%%%%%%%%%%%%%%%%%%%%% mise en page de l'index%%%%%%%%%%%%%%%%%%%%%%%%%%%%%%%%%%%%
\usepackage{multicol}
\makeatletter
\renewenvironment{theindex}
	{\if@twocolumn
    	\@restonecolfalse
       	\else
          \@restonecoltrue
       \fi
       \setlength{\columnseprule}{0pt}
       \setlength{\columnsep}{35pt}
      \begin{multicols*}{5}[\section*{\indexname}]
       \markboth{\MakeUppercase\indexname}%
                {\MakeUppercase\indexname}%
       \thispagestyle{plain}
       \setlength{\parindent}{0pt}
       \setlength{\parskip}{0pt plus 0.3pt}
       \relax
       \let\item\@idxitem}%
    {\end{multicols*}\if@restonecol\onecolumn\else\clearpage\fi}
\makeatother

\makeatletter
\renewcommand\@idxitem{\par\hangindent 40\p@}
\renewcommand\subitem{\@idxitem \hspace*{20\p@}}
\renewcommand\subsubitem{\@idxitem \hspace*{30\p@}}
\renewcommand\indexspace{\par \vskip 10\p@ \@plus5\p@ \@minus3\p@\relax}

\renewcommand{\contentsname}{Table des Matières}

\renewcommand\section{\@startsection {section}{1}{\z@}%
	{-3.5ex \@plus -1ex \@minus -.2ex}%
	{2.3ex}%
	{\reset@font\Large\bfseries}}
	

\makeatother
%%%%%%%%%%%%%%%%%%%%%%%%%%%%%% mise en page de l'index%%%%%%%%%%%%%%%%%%%%%%%%%%%%%%%%%%%%


\makeindex
\newcommand{\grasindex}[1]{\textbf{#1} \index{#1\textbf}}
\newcommand{\bu}[0]{\ifadvanced \else \hspace*{-1mm}$\bigstar$ \hspace*{0.15mm}\fi}
\newcommand{\adv}[1]{\ifadvanced #1 \fi}




\author{\textbf{Adrien Taudiere}}
\title{Fiche de commande R}

\date\today




%%%%%%%%%%%%%%%%%%%%%%%%%% Page de garde %%%%%%%%%%%%%%%%%%%%%%%%%%%%%%%%%%%%%%%%%%%%%%%%%%%

\begin{document}

\begin{multicols*}{3}

\maketitle

\tableofcontents
\rule{8cm}{1.5pt}

\setlength{\parindent}{-0.3cm}

\section{Formalisme} 

\textbf{a} et \textbf{b} sont des matrices (\grasindex{matrix}() pour 2 dimensions ou bien des \grasindex{array}() pour plus de 2 dimensions)

\textbf{X} et \textbf{Y} sont des vecteurs (\grasindex{vector}())

\textbf{x} et \textbf{y} sont des nombres 

\textbf{liste} est une liste c'est à dire une groupe d'objet pouvant être différents (\grasindex{list}())

\textbf{fact} est un facteur (\grasindex{factor}())

\textbf{df} est un data.frame (liste d'éléments de même longueur; \grasindex{data.frame}()) 

\textbf{\textbar{}} correspond à 'ou' dans R et dans cette fiche 

\textbf{m} est un modèle

\textbf{obj} est un objet quelconques

\textbf{T} = TRUE; \textbf{F} = FALSE

\textbf{élt} élément

\textbf{ddl} degré de liberté

\textbf{µ} moyenne

\textbf{physeq} object de classe phyloseq

\textbf{cond} condition


\section{Importation et gestion des données avec le package phyloseq}


\subsection{Importation}

\grasindex{import\_biom}()

\grasindex{import\_qiime}()

\grasindex{merge\_phyloseq}()


\subsection{Gestion des objets de classe phyloseq}

\grasindex{prune\_samples}(physeq, cond)

\grasindex{prune\_taxa}(physeq, cond)


\subsection{Construction d'un arbre phylogénétique à partir des séquences de références}

\grasindex{AlignSeqs}(\grasindex{as.DNAbin}(physeq@refseqs))

\grasindex{StaggerAlignment}

\grasindex{BrowseSeqs}

\grasindex{DistanceMatrix}

\grasindex{upgma} \grasindex{NJ}

\section{Exploration simple des données}

\grasindex{sample\_sums}(physeq)

\grasindex{taxa\_sums}(physeq)


\grasindex{sankey\_phyloseq}(physeq)

\grasindex{treemap}(df)

\section{Diversité alpha}

\grasindex{plot\_richness}(physeq)

\grasindex{accu\_plot}(physeq)

\grasindex{renyi}(physeq, hill = T) On peut utiliser les nombres de Hill dans un modèle linéaire \grasindex{lm}

\grasindex{estimate_richness}

\section{Diversité beta}

\subsection{Figures}

\grasindex{venn\_phyloseq}(physeq) Voire les alternatives moins clefs en main : \grasindex{venneuler} du package éponyme et \grasindex{venn.plot} du package \grasindex{VennDiagram}

\grasindex{plot\_net}

\grasindex{plot\_heatmap}

\grasindex{otu\_circle}


\subsection{Ordinations}

\grasindex{ordinate}(physeq) La fonction graphique associés est \grasindex{plot\_ordination}.

\subsection{Permanova}

\grasindex{adonis}(physeq@otu\_table) Package \grasindex{vegan}



\subsection{Abondance différentielle des OTUs}

	\subsubsection{Package DESeq2}

\grasindex{phyloseq\_to\_deseq2}

\grasindex{DESeq2}

\grasindex{results}

\grasindex{plot\_deseq2\_phyloseq}

	\subsubsection{Package edgeR}

\grasindex{plot\_edgeR\_phyloseq}

	\subsubsection{Package multtest}

\grasindex{mt}

\grasindex{plot\_mt}

	\subsubsection{Package mvabund}

\grasindex{manyglm}

\grasindex{anova.manyglm}

\grasindex{summary.manyglm}





\end{multicols*}

\printindex
\addcontentsline{toc}{subsection}{Index}


\end{document}
